\documentclass[11pt,a4paper]{moderncv}
\moderncvtheme[blue]{classic}
\renewcommand\familydefault{\sfdefault}

\renewcommand*{\namefont}{\fontsize{24}{29}\mdseries\upshape}


%http://latexcolor.com/
\definecolor{arsenic}{rgb}{0.23, 0.27, 0.29}
\definecolor{cornflowerblue}{rgb}{0.39, 0.58, 0.93}
\definecolor{darkcerulean}{rgb}{0.03, 0.27, 0.49}
\AfterPreamble{\hypersetup{
		   colorlinks = true,
            linkcolor = blue,
            urlcolor  = darkcerulean,
            citecolor = blue,
            anchorcolor = blue
}}
\usepackage[utf8]{inputenc}
\usepackage[top=1cm, bottom=1cm, left=1cm, right=1.2cm]{geometry}
\setlength{\hintscolumnwidth}{2.25cm}

\firstname{Guillaume}
\familyname{Pressiat}

\title{Statisticien appliqué}
\address{3 rue Delbet}{75014 Paris}
\email{guillaume.pressiat@gmail.com}
\extrainfo{Âge : 30 ans}
\mobile{06 71 15 28 28}

\photo[40pt][0.4pt]{photo.png}

\begin{document}
\maketitle


\section{Expériences professionnelles}
\linespread{1.1}
\cventry{2014 -- 2019}{AP-HP}{Assistance Publique - Hôpitaux de Paris}{Paris}{}{
\begin{itemize}
\item Au sein du département d'information médicale (DIM) du Siège de l’AP-HP
\item Études thématiques : cancérologie, parts d'activités régionales, rapprochement inter-champs
\item  Envois PMSI : consolidation de l'entité juridique AP-HP à partir de ses entités géographiques
\item  Processus automatisés du management des données à la production d'indicateurs
\item Cartographies (géolocalisation, cartes dynamiques, cartes choroplèthes)\href{https://guillaumepressiat.github.io/finess_etalab/rmd/import_etalab.html}{*}
\item Parcours de patients par pathologie (consultations / hospitalisations, délais, flux avec sunburst, TramineR)
\item Communication et poster congrès Emois : Approche géo-populationnelle de la précarité à partir du PMSI et de données Insee (indicateur FDep) ; Chimiothérapie et décès pour cancer métastique
\item Algorithmes de machine learning et de traitement automatisé du langage appliqués à la classification de compte-rendus hospitaliers (CRH, CRO) pour aider au codage PMSI
\item Développements : \href{https://github.com/GuillaumePressiat/}{lien github}
\begin{itemize}
\item \href{https://guillaumepressiat.github.io/}{pmeasyr} : données PMSI avec R - \href{https://guillaumepressiat.github.io/nomensland}{nomensland} : nomenclatures et référentiels PMSI - voir cette \href{https://guillaumepressiat.github.io/miscellany/connexes/}{représentation} pour plus d'informations
\item Applications et api pour diffuser les référentiels et méthodes du PMSI dans l’institution
\end{itemize}
\end{itemize}}

\cventry{2013\hspace{0.625cm}}{IRi}{Information Resources, Inc.}{Chambourcy}{}{
\begin{itemize}
\item Études de faisabilité et réalisation d’études statistiques - marketing grande distribution
\item Coordination d’études délocalisées en Grèce (communication en anglais)
\item Projet SAS pour développer un nouveau type d’étude en Grèce
\end{itemize}}
\cventry{2012\hspace{0.625cm}}{CNPS-CRMM}{Centre de Neurosciences de Paris Sud (Orsay) et Centre de Recherche sur les Mammifères Marins (La Rochelle)}{Stage Recherche de Master 2}{}{
\begin{itemize}
\item Estimation du nombre de baleines bleues dans l’océan Indien à partir de données acoustiques
\item Modélisation du rythme des vocalises de baleines bleues à l’aide d’un modèle de segmentation gaussien 
\end{itemize}}

\linespread{1}
\section{Programmation et informatique}

\cvitem{Programmes}{R, Python, SAS, SQL, Git, Hive, Spark}
\cvitem{Reporting}{LaTeX, markdown, HTML, Javascript, Jupyter}
\cvitem{Bureautique}{Suite Office (Excel, Word, Powerpoint), Adobe CS5}


\section{Études}
\cventry{2011 -- 2012}{Master 2 Mathématiques appliquées}{Paris Descartes}{}{}{}{}
\cventry{2009 -- 2011}{Master 1 Mathématiques fondamentales}{Paris Diderot}{}{}{}
\cventry{2006 -- 2009}{Licence Mathématiques Informatique et Sciences de la Matière}{Limoges (87)}{}{}{}

\section{Langues}
\cvitem{Anglais}{Professionel : lecture d’articles, enseignements suivis en anglais}
\cvitem{Allemand}{Connaissances de base}

\section{Expériences annexes, loisirs}
\cvitem{2007 -- 2008}{Cours particuliers de mathématiques à des élèves de collège et de lycée}{}{}
\cvitem{2005 -- 2012}{Travail sur une exploitation agricole en période estivale}{}{}
\cvitem{Loisirs}{Guitare, style finger-picking, folk, rock, blues; Sports : Vélo (route, cyclosport-randonée), pêche aux leurres; Lecture)} 

%\section{Loisirs}
%\cvitem{}{Guitare, style finger-picking, folk, rock, blues; Sports : Vélo (route, cyclosport-randonée), pêche aux leurres; Lecture)} 
%\cvitem{}{Sports : Vélo (route, cyclosport-randonée), pêche aux leurres}
%\cvitem{}{Lecture}


\end{document}
